%% Tentando descobrir como fazer referência e citações numéricas
%% ABNT.
%% Teste para o projeto abntex/limarka

\documentclass{abntex2}

\usepackage{etoolbox}
\usepackage[numbers,round,comma]{natbib}
%\usepackage[num]{abntex2cite}

\makeatletter % Reference list option change
\renewcommand\@biblabel[1]{#1} % from [1] to 1
\makeatother %


\begin{document}


\AtEndEnvironment{thebibliography}{
% all your extra bibitems go here
\bibitem{CRETELLA1992} CRETELLA JUNIOR, Jose. {\em Do impeachment no direito brasileiro}.
[São Paulo]: R. dos Tribunais, 1992. p. 107. mhas jdhgasjd kjhsagdkjhgas kdjsa.
}

De acordo com as novas tendências da jurisprudência brasileira \cite{CRETELLA1992}, é facultado ao
magistrado decidir sobre a matéria.


Todos os índices coletados para a região escolhida foram analisados minuciosamente 2 .
Na lista de referências:

 
BOLETIM ESTATÍSTICO [da] Rede Ferroviária Federal. Rio de
Janeiro, 1965. p. 20.


Cite from above \cite{extra1}.

\AtEndEnvironment{thebibliography}{
% all your extra bibitems go here
\bibitem{extra1}SOBRENOME, N. {\emph Titulo}: Subtítulo. 20XX.
}


%\bibliographystyle{plainnat}

\bibliographystyle{vancouver}
\bibliography{master.bib} 

\end{document}
